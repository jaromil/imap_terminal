\documentclass[titlepage]{article}
\usepackage[utf8]{inputenc}
\usepackage{graphicx}
\usepackage{longtable}
\usepackage[russian]{babel}
\usepackage{wasysym}
\usepackage{longtable}
\usepackage{listings}
\usepackage[colorlinks=true]{hyperref}
\title{IMAP terminal program}
\author{Aleksei Malnev}

\begin{document}
\section{Building}
To build in MS Visual Studio use \textbf{imap\textunderscore terminal.sln} file provided with the source. 
To build in Linux use CMake build system. Create a separate build directory, cd into it,
configure and build \textbf{imap\textunderscore terminal} like this:
\begin{lstlisting}
	mkdir path/to/build/dir
	cd path/to/build/dir
	cmake path/to/source/dir
	cmake --build .
\end{lstlisting}
\textbf{imap\textunderscore terminal} depends on libcurl to exchange data with IMAP servers.
For Windows the most recent stable version of libcurl is pre-built and \textbf{libcurl.dll} is 
put in the same directory with \textbf{imap\textunderscore terminal.exe} automatically by 
MS Visual Studio build system. In Linux
libcurl is available for installation from repositories in all modern distributions. For example,
in Ubuntu-based distros install libcurl by runnning
\begin{lstlisting}
123
\end{lstlisting}
This must be done before trying to build \textbf{imap\textunderscore terminal}.

\section{Using}
imap\textunderscore terminal has the following command line syntax:
\begin{lstlisting}
imap_terminal -h|-host=<...> [-p|-port=<...>] -u|-user=<...> 
	[-P|-pass=<...>] [-s|-ssl]
\end{lstlisting}
Parameters \textbf{host, port, user, pass} are self-explanatory. Parameter \textbf{-ssl}  indicates
that SSL should be used to connect. Parameters \textbf{host} and \textbf{user} are required, 
everything else is optional. If \textbf{port} parameter is omitted then 993 is used as a default. If
\textbf{-ssl} switch is not given on the command line, imap\textunderscore terminal will try to connect
without using SSL. For example:
\begin{lstlisting}
imap_terminal -h=imap.gmail.com -p=993 -u=me@gmail.com 
	-P=mypass -ssl
\end{lstlisting}
will connect to the Gmail service. When connection is established, an interactive shell-like interface
is presented. This interface can be used to manipulate messages and directories in the user's mailbox.
The following commands are implemented.

\subsection{Working with directory tree}
Use \textbf{ls, cd, pwd, mkdir, rmdir} commands to work with directories. 
\begin{enumerate}
	\item
	\textbf{cd} command changes the current directory. Accepts both relative and absolute paths. 
	Understands "." and ".." special dirs. Example:
	\begin{lstlisting}
		cd INBOX/subdir1
		cd "../../[Gmail]/Sent Mail"
	\end{lstlisting}
	
	\item
	\textbf{pwd} command prints the current directory
	
	\item
	\textbf{mkdir, rmdir} commands create and delete subdirectories inside the current directory. Example:
	\begin{lstlisting}
		$ mkdir subdir1
		$ cd subdir1
		$ pwd
		/subdir1
		$
	\end{lstlisting}
	
	\item
	Use \textbf{ls} command without arguments to list the contents of the current directory. It will
	start by printing subdirectories in the current directory
\end{enumerate}

\subsection{Working with messages}
Use \textbf{ls, rm, mv, head} commands to work with messages. These commands share the common syntax:
\begin{lstlisting}
	ls|head|rm|mv [-s|-subject=<...>] [-t|-to=<...>] 
		[-f|-from=<...>] [message_uid] [destination]
\end{lstlisting}
\textbf{-subject, -to, -from} options or their short synonims \textbf{-s, -t, -f} can be used
to filter messages in the current directory to operate on. Only messages matching the given 
criteria will be operated on. To do somaething to the individual message supply message UID. 
[destination] parameter is required for mv command. Examples:
\begin{lstlisting}
	ls 3
	ls -from=me@gmail.com
	ls -s=<<a message>> -from=me@gmail.com
	ls
\end{lstlisting}
First command lists message with UID=3. Second command lists all messages from me@gmail.com
in the current directory. Third command lists all messages in the current directory from
me@gmail.com \textbf{AND} having subject <<a message>>. Fourth command lists everything in
the current directory.

\par
\textbf{ls, rm, mv, head} commands always operate on a subset of messages in the current 
directory. By default the scope of these commands is limited to 20 most recent messages
in the current directory. To change this value use \textbf{limit} command (described below).

\begin{enumerate}
	\item
	\textbf{ls} command can be used to list messages in the current directory.
		
	\item
	\textbf{mv} command can be used to move messages between subdirectories. Usage:
	\begin{lstlisting}
	mv <what> <where>
	\end{lstlisting}
	<what> can be a message UID or a set of selection options like -s etc. as described above.
	<where> is an absolute or relative path to the destination directory 
	
	\item
	\textbf{rm} command removes messages from the current directory. Be careful to issue 
	\textbf{rm} - it will remove all messages in the current directory.
	
	\item
	\textbf{head} command outputs the beginning of a message body
\end{enumerate}

\subsection{Other commands}
\begin{enumerate}
	\item
	\textbf{limit} command is used to limit the scope of message commands described in section 2.2.
	Usage:
	\begin{lstlisting}
	limit [N]
	\end{lstlisting}
	\textbf{limit} without arguments prints the current value. \textbf{limit N} updates the current
	value to \textbf{N}.
	
	\item
	\textbf{exit} or \textbf{quit} commands make imap\textunderscore terminal close current
	session and terminate.
	
	\item
	\textbf{whoami} command prints the username currently used to authenticate to IMAP server.
\end{enumerate}

\subsection{Example session}
\begin{longtable}{|p{.3\textwidth}|p{.4\textwidth}|p{.25\textwidth}|}
	  \hline
	  \textbf{Command} & \textbf{Output} & \textbf{Explanation} \\
	  
	  \hline
	  imap\textunderscore terminal \newline -h=imap.gmail.com \newline -p=993 \newline
	  -u=me@gmail.com \newline -P=mypass -ssl & none & Run the program and connect to Gmail \\
	  
	  \hline
	  \$ pwd & / &
	  show the current directory \\
	  
	  \hline
	  \$ ls & d    INBOX \newline d    [Gmail] \newline d    test\textunderscore label &
	  list the contents of the current directory \\
	  
	  \hline
	  \$ cd INBOX &  &
	  change the current directory to \textbf{/INBOX} \\
	  
	  \hline
	  \$ pwd & /INBOX &
	  show the current directory \\
	  
	  \hline
	  \$ ls & 1  From:... Subject:... &
	  list the contents of the current directory. Listing shows that the message with UID=1 exists\\
	  
	  \hline
	  \$ mv \newline -s=... \newline <<../test\textunderscore label>> &  &
	  Move the message with given subject to the \textbf{../test\textunderscore label} directory\\
	  
	  \hline
	  \$ cd ../test\textunderscore label &  &
	  change the current directory to \textbf{/test\textunderscore label} \\
	  
	  \hline
	  \$ ls & d  test\textunderscore sublabel\textunderscore 1 \newline
		d  test\textunderscore sublabel\textunderscore 2 \newline
		3  From:..  Subject:\textbf{test1} \newline
		2  From:..  Subject:... \newline
		1  From:..  Subject:... &
	  List the contents of the current directory. Ensure that previous \textbf{mv} command succeeded\\
	  
	  \hline
	  
\end{longtable}

\end{document}